\chapter*{内容梗概}


%It has been anticipated that a so-called "soft error"
%causes serious problems even in a combinatorial logic circuits consisting of the near-future generation of transistors.
近い将来,トランジスタによって構成される組み合わせ回路では,
動作中に発生する可能性のあるソフトエラーが深刻な問題となることが予想されている.
%Thus it is very important to develop efficient methods for the reliability analysis of logic circuits.
したがって,論理回路の信頼性を効率的に評価する手法の開発は,
回路の設計において非常に重要となってきている.

%There is an efficient reliability analysis method based on Probabilistic Transfer Matrix (PTM),
論理回路の信頼性を評価するための既存の手法としては,Probabilistic Transfer Matrix (PTM) を用いた手法が存在する.
%but the necessary peak memory would be very huge in the worst case.
しかし,この手法は,最悪の場合,非常に大きなメモリが必要となる.
%This paper proposes another approach that uses much less memory even in the worst case.
そこで,本研究では,既存手法とは異なるアプローチを用いてメモリの使用量を削減する.
%Our method carefully considers to avoid essentially the same computation
%as much as possible to reduce the computational time.
\cyannout{このアルゴリズムは,各入力に対して出力が正しく得られる確率を用いて,ゲート数に比例した空間計算量で計算する.
計算途中に発生する同じ計算を減らすことによって,計算時間を減らす工夫も行われている.}

%Our preliminary experimental results show that the memory usage would be less than a ten-thousandth
%in the best case compared to the state-of-the-art PTM-based simulator.
本手法は,最新の PTM を用いた手法の結果と比較実験を行った.
その結果,最も効果的なケースでは,メモリの使用量を 2 万分の 1 程度に削減することに成功した.

\ifnum\NGCcolor=0
\section*{修正色}

\begin{itemize}
  \item \redout{先生}
  \item \blueout{ikさん}
  \item \greenout{TKGさん}
  \item \cyannout{自分}
\end{itemize}
\fi
