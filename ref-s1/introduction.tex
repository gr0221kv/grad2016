\chapter{はじめに}\label{sec:introduction}

%\paragraph{(どのような分野の研究か,その背景について説明する)}
集積回路の縮小や動作電圧の低下に伴い,
放射線がゲートに与える影響は,無視できないものとなっている.
このような影響によって一時的に発生するエラーは,ソフトエラーと呼ばれる~\cite{Baumann:2005:SEA:1069593.1069707}.
近年,ソフトエラーの発生が顕著になってきたことで,
このエラーの発生を考慮した設計が求められるようになった.
そのため,回路の正確な信頼性を,できるだけ効率的に評価する手法が必要となってきている.

%\paragraph{(その分野の従来の研究状況について説明する)}
ソフトエラーの発生は,必ずしも回路の外部出力に影響を与えるとは限らない.
例えば,2 入力のANDゲートに本来 $(00)_2$ の入力を与えるべき状況において,
二つのうち一つの入力が反転したとしても最終的な結果が変化することはない.
このように,ゲートの論理によってソフトエラーの伝搬を防ぎ止める効果を論理マスクと呼ぶ.
一方で,入力が $(11)_2$ の際には,どちらの入力も反転することは許されない.
この例のように,論理マスクが発生するかどうかは,ゲートの論理とその入力によって決定される.
したがって,回路の出力が最終的に誤りとなる確率は,ゲートの組み合わせとその回路に与えられる入力に依存する.
すなわち,回路の入力数 $n$ に対して $2^n$ パターンの場合の回路の故障確率を考える必要がある.
そこで, $n$ 入力の回路を評価する際には,
$2^n$ パターンについての信頼性を考慮しなくてはならない.
このような計算は, $n$ が大きくなると困難であるため,
通常はモンテカルロ法を用いたフォールト挿入テストが用いられる.
%ik: 「あるいは」で繋がずに一文で並列したら良いのでは
% -> 考えうる一般的な手法と解析的な手法で分けたので,同じように扱うのは微妙?
あるいは解析的な手法として,
Probabilistic Transfer Matrix (PTM)~\cite{Krishnaswamy:2008:PTM:1297666.1297674}
を用いた手法が存在する.
% 蛇足っぽい
一方,これらとは異なる考え方として,論理マスクについて見積る手法も提案されている~\cite{松永裕介:2008-09-29}.

%\paragraph{(そして,何が解決すべき問題(本論文で扱った問題)かを説明)}
PTM を用いた手法では,ゲートや部分回路の入出力の対応を行列で表し,
その行列同士の積やクロネッカー積を用いて目的とする回路に対応した行列を得る.
この時,行列のサイズは,クロネッカー積によって指数的に増加する.
そこで,既存手法では Algebraic Decision Diagram (ADD)~\cite{580054}
を用いた実装を行い,空間計算量の問題に対処している.
ADD は,論理関数を表現する Binary Decision Diagram~\cite{BLTJ:BLTJ1585} から派生したデータ構造の一つであり,
行列やベクトルをコンパクトに管理しながら,演算を行うことができる.
しかしながら, ADD の終端ノードは,
行列中の保持しなくてはならない値の種類数だけ存在する.
したがって,最悪の場合は指数的な量のメモリが必要となり,
大規模な回路に対する PTM を生成するためには十分ではない.

%\paragraph{(どのようなアイデアで解決したか,キーアイデアを少しだけ披露)}
そこで本論文では,空間計算量を削減した評価手法を提案する.
提案手法では,既存手法の最終的な評価式を変形し,
各入力に対する出力が正しく得られる確率を,無閉路有向グラフとして計算する.
\redout{この手法によって,}ゲートの数に比例した量のメモリで計算することができる.%\todo{実験後に見直しが必要}
特に大幅にメモリの使用量を削減したケースとして,
PTM を使った手法と比べて約2万分の1となったケースも存在する.
この手法は,再収斂が存在する回路では正しい結果が得られないが,
その誤差は,疑似乱数を用いたシミュレーション結果と比較して 10\% 程度に抑えられている.
さらに,時間計算量と正確性の向上に向け,
本論文では数式処理によって計算する手法も提案する.
この手法は,ファンアウト・ポイントで分割し,
一つの部分回路を関数とみなして計算を行う.
このとき,同一の部分回路を表す関数同士の計算には,
特殊な演算規則を適用するようにすることで,再収斂に対する問題を回避する.

%\paragraph{(後の構成について)}
以下では,まず,第~\ref{sec:ptm} 章で既存手法として PTM について解説する.
次に,第~\ref{sec:proposed}~章では空間計算量を削減した評価手法とその改良について提案する.% 「改良について」は微妙
その後,第~\ref{sec:experiment}~章で実験と考察を述べ,
第~\ref{sec:conclusion} 章で本研究のまとめと今後の課題について述べる.
