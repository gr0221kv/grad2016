\documentclass[a4j,12pt,dvipdfmx]{jreport}

%%%%%%%%%%%%%%%%%%%%%%%%%%%%%%%%%%%%%%%%%%%%%%%%%%%%%%%%%%%%%%%
% 1. texの冒頭に\usepackage{NGC}と記述してください.
%
% 2. 文字列の色替えをキャンセルしたい(文字を黒くしたい)場合は,
%    NGC.styの中身を編集してください.
%    編集手順はNGC.styに書いてあります.
%%%%%%%%%%%%%%%%%%%%%%%%%%%%%%%%%%%%%%%%%%%%%%%%%%%%%%%%%%%%%%%

\usepackage{NGC}
\usepackage{algorithmic,algorithm}
\usepackage{amsmath}
\usepackage{amssymb}
\usepackage{amsthm}
\usepackage{bm}
\usepackage{graphicx}
\usepackage{todonotes}
\usepackage{url}

% 図・表がページ全体を占めても良い割合を少し増やす
\renewcommand{\topfraction}{0.7} % 上部で占める最大の割合
\renewcommand{\bottomfraction}{0.7} % 下部で占める最大の割合
\renewcommand{\textfraction}{0.3} % テキストが占める最小割合
\renewcommand{\floatpagefraction}{0.7} % 単独になるときの最小割合

\renewcommand{\arraystretch}{.9}

\title{ユニタリ行列の座標化に基づいた量子ゲートの分解の効率化}
\author{PHAM TRUNG TIEN}
\stdnum{6611140060-2}

\begin{document}

\maketitle

\newpage
\thispagestyle{empty}
\pagenumbering{roman}
\setcounter{page}{1}
\chapter*{内容梗概}


%It has been anticipated that a so-called "soft error"
%causes serious problems even in a combinatorial logic circuits consisting of the near-future generation of transistors.
近い将来,トランジスタによって構成される組み合わせ回路では,
動作中に発生する可能性のあるソフトエラーが深刻な問題となることが予想されている.
%Thus it is very important to develop efficient methods for the reliability analysis of logic circuits.
したがって,論理回路の信頼性を効率的に評価する手法の開発は,
回路の設計において非常に重要となってきている.

%There is an efficient reliability analysis method based on Probabilistic Transfer Matrix (PTM),
論理回路の信頼性を評価するための既存の手法としては,Probabilistic Transfer Matrix (PTM) を用いた手法が存在する.
%but the necessary peak memory would be very huge in the worst case.
しかし,この手法は,最悪の場合,非常に大きなメモリが必要となる.
%This paper proposes another approach that uses much less memory even in the worst case.
そこで,本研究では,既存手法とは異なるアプローチを用いてメモリの使用量を削減する.
%Our method carefully considers to avoid essentially the same computation
%as much as possible to reduce the computational time.
\cyannout{このアルゴリズムは,各入力に対して出力が正しく得られる確率を用いて,ゲート数に比例した空間計算量で計算する.
計算途中に発生する同じ計算を減らすことによって,計算時間を減らす工夫も行われている.}

%Our preliminary experimental results show that the memory usage would be less than a ten-thousandth
%in the best case compared to the state-of-the-art PTM-based simulator.
本手法は,最新の PTM を用いた手法の結果と比較実験を行った.
その結果,最も効果的なケースでは,メモリの使用量を 2 万分の 1 程度に削減することに成功した.

\ifnum\NGCcolor=0
\section*{修正色}

\begin{itemize}
  \item \redout{先生}
  \item \blueout{ikさん}
  \item \greenout{TKGさん}
  \item \cyannout{自分}
\end{itemize}
\fi

\newpage
\tableofcontents
\listoffigures
\listoftables

\newpage
\pagenumbering{arabic}
\chapter{はじめに}\label{chap:introduction}
\chapter{背景}\label{chap:background}
\section{量子計算}\label{sec:quantum_computing}
\section{量子回路と量子ゲート}\label{sec:quantum_circuit}
\section{量子ゲート分解とユニタリ行列の分解問題}\label{sec:quantum_circuit}
\section{部分的な分解の組み合わせによるアプローチ}\label{sec:related_works}
\subsection{アプローチの概要}
分解結果をビルディングブロックから構成するステージには、Brute-forceの従来のアルゴリズムは全ての行列組み合わせを生成し、検証する必要がある。アルゴリズムの詳細を以下に述べる。

なお、$D$は$(d \times d)$行列の距離関数である。$D$の例としては、$D(V, W) \equiv \sqrt{\frac{d - |Trace(V*W^{\dagger})|}{d}}$、量子計算の研究ではよく使われる距離関数である。

\subsection{一般のシナリオ}
\subsection{Selingerらの手法への適用するシナリオ}
\chapter{問題定義}\label{chap:problem_def}
\section{メイン問題:}\label{sec:main_problem}
両方のシナリオに組み合わせ数に伴う計算コストの問題が存在
\section{サブ問題:}\label{sec:sub_problem}
一般のシナリオの場合、初期探索空間の生成、記憶するコスト問題
\chapter{提案手法}\label{chap:proposal}
\section{ユニタリ行列の座標化による効率化手法}\label{sec:propsal1}
\subsection{ユニタリ行列の座標の概念}
ユニタリ $\rightarrow$ エルミート$\rightarrow$実数体上のベクトルとのマッピング
\subsection{ユニタリ行列の積と座標の和の関係}
- ユニタリ行列の積と座標の和の誤差に関する不等式を
\subsection{提案した問題への繋がり}
- 全組み合わせ$\rightarrow$ソート済配列の探索へ
- ベクトル演算子の条件
\subsection{アルゴリズムの詳細}
- 条件を満たすベクトル演算子の例
- 仮ソースコード
\subsection{考察}
- 提案手法はNearIdentity以外に適用ようできるための添削

\chapter{実験結果}\label{chap:experiment}
\section{一般のシナリオ向けの実験}
\subsection{2x2行列の分解}
\subsection{4x4行列の分解}
\section{Selingerらの手法への適用シナリオ向けの実験}

\chapter{おわりに}\label{sec:conclusion}

% 本論文の概要と特徴
%In this work, we proposed the method to analyze the reliability of logic circuits.
%This method is efficient in memory usage by using directed acyclic graphs,
%and furthermore, Gray Code improves the execution time.
本研究では,組み合わせ回路における信頼性の評価方法について提案した.
提案手法は,無閉路有向グラフの利用によって,メモリの使用量を削減し,
グレイコードによって計算時間の面でも効率化を行った.
ただし,回路内に再収斂が存在する場合,この手法では正確な確率を計算することはできない.
そこでさらに,数式として計算し補正する手法と,そのための数式処理システムを提案した.

% 得られた成果
%The method does not calculate the correct probabilities if the circuit has the recomputation.
%However, this reconvergence do not heavily affect the results.
%According to our experiment,
%the memory usage is less than a ten-thousandth in the best case compared to the existing method
%and the calculated probabilities have accuracy within about 10\%.
実験結果の結果,数値として計算を行う提案手法は,
Probabilistic Transfer Matrix (PTM) を用いた既存手法に比べて,
メモリの使用量を 2 万分の 1 に抑えられるケースも存在する.
また,再収斂による誤差は,シミュレーションとの比較によれば,ほとんどのケースで 10\% 以内に抑えられた.
したがって,信頼性の評価に対して極めて深刻な誤差が発生するとは限らない.

% それから得られる最終結論

%このヒューリスティクスは、再集連を含まない場合に限り、正確な答えを出す。
%提案手法は、既存の厳密解を得る手法と比べてメモリの面で効率的であり、
%既存の入力パターンのサンプリングを行う場合と違い、
%全ての入力パターンに対して評価を行うことができる。
%あるいは、このヒューリスティクスに対して、
%さらに入力のサンプリングを行うことも可能である。

% 残された課題

既存手法では,指数的に増加する計算時間に対処するために,
いくつかの入力をサンプリングしたヒューリスティクスも開発されている.
提案手法についても,同様のヒューリスティクスが考えられる.
このとき,\ref{sec:prop:deform:reuse} を考慮すると,グレイコードによって,
1 ビット違いの連続した入力をサンプリングした方が効率が良い.
しかし,このサンプリング方法は,ランダムな入力を用いた場合に比べて偏りが生じるかもしれない.
サンプリングの方法によって,どの程度の偏りが生じるかは,検討すべき課題である.

提案手法は,メモリの使用量を削減したことで,分散コンピューティングにも容易に応用できる.
並列計算は,一般的にデータの転送時間がボトルネックとなりやすい.
しかし,提案手法の計算を入力パターンで分割し,ぞれぞれの入力パターンについて同時に計算する手法を考えると,
グラフ構造や数式のデータ構造を,各ユニットへ配布するだけでベクトル ${\bm s}$ を計算することができる.
さらに,${\it fidelity}$ の $l_1$-ノルムは,いわゆる並列リダクションで処理できる.
したがって,式~(\ref{eq:deformed}) の ${\bm v} \cdot {\bm s}$ の計算は,
MapReduce~\cite{mapreduce} と呼ばれるプログラミングモデルを適用することができると言える.
ベクトル ${\bm s}$ の計算と,ベクトル ${\bm v}$ との各要素同士の積を求めるまでが Map 操作に相当し,
それらの合計を求めるのが Reduce 操作に相当する.
分散コンピューティングで成功したモデルの一つである MapReduce との組み合わせによって,
設計時における信頼性の計算はさらなる高速化が期待できる.

%今後の課題としては、速度と正確性の改善である。
%The future study will focus on the speed and accuracy.
%Using a hash table of the sub-circuits may accelerate this method.
%By dividing a whole circuit and recomposing, we can omit the calculation of the same sub-circuit.
%Then the division and composition can solve the reconvergent problem
%when the input side of the sub-circuits do not contains many reconvergence.
%In addition, the accuracy of the results may rely on their topological distance.
%In that case, we may need to consider only gates near the reconvergent point to correct the error
%because it is anticipated that the effect of a distant reconvergent point would be trivial.


\chapter*{謝辞}

「このロボットを原発事故の現場に持っていくことはできないんですか?」\par
「あぁ,それはね,電磁波や放射線の影響で誤作動しちゃうんだよ…」\\
何年か前のうろ覚えな記憶ではありますが,RoboCup Japan Open の会場でこのような会話を耳にしました.
こうしてソフトエラーをテーマにした修士論文を書いてから思い返すと,
僅かながら運命的なつながりを感じる部分があります.
本研究のきっかけを与え,多大なるご指導をして頂きました,
立命館大学 情報理工学部 山下茂教授に深く感謝しお礼申し上げます.

次世代コンピューティング研究室の皆様に深く感謝申し上げます.
ユニークなイベントの数々は,他では味わえない貴重な経験になりました.
あらゆる立場から接していただき,将来の人生を豊かにするための様々なヒント得ることができたと思います.
共に過した多くの時間は,決して忘れることないでしょう.

情報理工学部プロジェクト団体や,立命館大学コンピュータクラブの皆様にもお礼申し上げます.
志の高い先輩や後輩,そして同期の方々に出会い,
成長の機会と経験を得ることができました.
今後とも活躍を期待し,応援したいと思います.

最後に,大学生活を支えていただき,育てていただいた両親にお礼申し上げます.

\begin{thebibliography}{9}

\bibitem{SK:2005}
C.~M. Dawson and M.~A. Nielsen.
\newblock The {S}olovay-{K}itaev algorithm.
\newblock {\em Quantum Information and Computation}, 6(1):81--95, 2006.

\bibitem{MA:2012}
Matthew Amy, Dmitri Maslov, Michele Mosca, and Martin Roetteler
\newblock A meet-in-the-middle algorithm for fast synthesis of depth-optimal quantum circuits.
\newblock {\em arXiv:1206.0758}, 2012.

\bibitem{VK:2013}
Vadym Kliuchnikov
\newblock Synthesis of unitaries with Clifford+T circuits.
\newblock {\tt arXiv:1306.3200}, 2013.

\end{thebibliography}


\appendix

\end{document}

